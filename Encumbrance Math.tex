\documentclass{article}

% Math support
\usepackage{amsmath}
\usepackage{amssymb}

% Better page width
\usepackage[margin=1in]{geometry}

\begin{document}

\section*{Full-System Carrying Capacity Function (Medium Size)}

Let $S$ be the Strength score (integer $\ge 1$).

The carrying capacity functions are:

\[
(\text{Light}(S),\ \text{Medium}(S),\ \text{Heavy}(S))
\]

\subsection*{Piece 1 — Strength less than or equal to 9 (table lookup)}

\[
(\text{Light}(S),\ \text{Medium}(S),\ \text{Heavy}(S)) = T(S)
\quad \text{for } S \le 9
\]

The lookup table:

\[
T(S)
=
\begin{array}{c|ccc}
S & \text{Light} & \text{Medium} & \text{Heavy} \\
\hline
1 & 3  & 6  & 10 \\
2 & 6  & 13 & 20 \\
3 & 10 & 20 & 30 \\
4 & 13 & 26 & 40 \\
5 & 16 & 33 & 50 \\
6 & 20 & 40 & 60 \\
7 & 23 & 46 & 70 \\
8 & 26 & 53 & 80 \\
9 & 30 & 60 & 90 \\
\end{array}
\]

\subsection*{Piece 2 — Strength greater than or equqal to 10 (formula)}

\[
d = S - 10, \qquad
k = \left\lfloor \frac{d}{5} \right\rfloor, \qquad
r = d \bmod 5
\]

\[
m(r) =
\begin{cases}
1.00 & r = 0 \\
1.17 & r = 1 \\
1.33 & r = 2 \\
1.50 & r = 3 \\
1.75 & r = 4
\end{cases}
\]

\[
\text{Heavy}(S)
=
\left\lfloor
100 \cdot 2^{k} \cdot m(r)
\right\rfloor
\]

\[
\text{Medium}(S)
=
\left\lfloor
\frac{2}{3}\,\text{Heavy}(S)
\right\rfloor
\]

\[
\text{Light}(S)
=
\left\lfloor
\frac{1}{3}\,\text{Heavy}(S)
\right\rfloor
\]

\subsection*{Full Combined Definition}

\[
(\text{Light}(S),\ \text{Medium}(S),\ \text{Heavy}(S))
=
\begin{cases}
T(S), & S \le 9 \\[6pt]
\left(
\left\lfloor \frac{1}{3}H(S) \right\rfloor,\ 
\left\lfloor \frac{2}{3}H(S) \right\rfloor,\ 
H(S)
\right),
& S \ge 10
\end{cases}
\]

\[
H(S)
=
\left\lfloor
100 \cdot 2^{\left\lfloor (S-10)/5 \right\rfloor}
\cdot
m\left( (S-10) \bmod 5 \right)
\right\rfloor
\]

\subsection*{Piece 3 — Size Scaling}

Each size category applies a uniform multiplier to all three load bands.
Define the size multiplier function $s(Z)$ by:

\[
s(Z) =
\begin{cases}
\frac{1}{8} & Z = \text{Fine}, \\
\frac{1}{4} & Z = \text{Diminutive}, \\
\frac{1}{2} & Z = \text{Tiny}, \\
\frac{3}{4} & Z = \text{Small}, \\
1           & Z = \text{Medium}, \\
2           & Z = \text{Large}, \\
4           & Z = \text{Huge}, \\
8           & Z = \text{Gargantuan}, \\
16          & Z = \text{Colossal}.
\end{cases}
\]

Let $\text{Light}(S)$, $\text{Medium}(S)$, and $\text{Heavy}(S)$ be the
carrying-capacity values computed for a Medium creature (from Pieces 1–2).
The size-adjusted load limits are then:

\[
\text{Light}_Z(S) =
\left\lfloor
\text{Light}(S) \cdot s(Z)
\right\rfloor,
\]

\[
\text{Medium}_Z(S) =
\left\lfloor
\text{Medium}(S) \cdot s(Z)
\right\rfloor,
\]

\[
\text{Heavy}_Z(S) =
\left\lfloor
\text{Heavy}(S) \cdot s(Z)
\right\rfloor.
\]


\end{document}
